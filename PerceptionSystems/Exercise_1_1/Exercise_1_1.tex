\documentclass[a4paper]{article}
\usepackage{amsmath}
\usepackage{hyperref}
\usepackage{enumitem}
\usepackage{graphicx}
\usepackage[utf8]{inputenc}
\usepackage[T1]{fontenc}
\usepackage{textcomp}
\usepackage{gensymb}

\title{Master on Robotics: Perception Systems - Exercise 1.1}
\date{28-10-2015}
\author{Juan Pedro López Cabrera}

\begin{document}
  \pagenumbering{gobble}
  \maketitle

  \newpage
  \pagenumbering{arabic}

  \section{Exercise 1}
If you have a mobile robot with wheels of radius R=0.4m, and it can run at maximum speed of 3m/s, compute how many pulses will
receive a counter if you use an encoder of 500ppr.
  \begin{itemize}
    \item r = 0.4m
    \item v = 3 m/s
    \item 500 pulses per revolution
  \end{itemize}


Given the relationship between angular velocity and linear velocity:

  \begin{equation}
  {\omega} = \dfrac {v}{r} = \dfrac {3}{0.4} = {7.5}\ {rad/s}
  \end{equation}


And given the relationship between angular velocity and frequency of rotation:

  \begin{equation}
  {f} = \dfrac {\omega}{2\pi} = \dfrac {7.5}{2\pi} \approx {1.19}\ {revolutions\ per\ second}
  \end{equation}


We can compute the number of pulses per second received by our encoder:

  \begin{equation}
  {result} = \dfrac {pulses}{revolution} \times \dfrac {revolutions}{second} = 500 \times 1.19 \approx {595}\ {pulses\ per\ second}
  \end{equation}

\end{document}
